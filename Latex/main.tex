% Arquivo LaTeX de exemplo de dissertação/tese a ser apresentados à CPG do IME-USP
% 
% Versão 5: Sex Mar  9 18:05:40 BRT 2012
%
% Criação: Jesús P. Mena-Chalco
% Revisão: Fabio Kon e Paulo Feofiloff
%  
% Obs: Leia previamente o texto do arquivo README.txt

\documentclass[12pt,oneside,a4paper]{book}

% ---------------------------------------------------------------------------- %
% Pacotes
\usepackage[OT1]{fontenc}
\usepackage[spanish]{babel}
\usepackage[utf8]{inputenc}
\usepackage{textcomp}
\usepackage[pdftex]{graphicx}           % usamos arquivos pdf/png como figuras
\usepackage{setspace}                   % espaçamento flexível
\usepackage{indentfirst}                % indentação do primeiro parágrafo
\usepackage{makeidx}                    % índice remissivo
\usepackage[nottoc]{tocbibind}          % acrescentamos a bibliografia/indice/conteudo no Table of Contents
\usepackage{courier}                    % usa o Adobe Courier no lugar de Computer Modern Typewriter
\usepackage{type1cm}                    % fontes realmente escaláveis
\usepackage{listings}                   % para formatar código-fonte (ex. em Java)
\usepackage{titletoc}
%\usepackage[bf,small,compact]{titlesec} % cabeçalhos dos títulos: menores e compactos
\usepackage{titlesec}
\usepackage[fixlanguage]{babelbib}
\usepackage[font=small,format=plain,labelfont=bf,up,textfont=it,up]{caption}
\usepackage[usenames,svgnames,dvipsnames]{xcolor}
\usepackage[a4paper,top=2.54cm,bottom=2.0cm,left=2.0cm,right=2.54cm]{geometry} % margens
%\usepackage[pdftex,plainpages=false,pdfpagelabels,pagebackref,colorlinks=true,citecolor=black,linkcolor=black,urlcolor=black,filecolor=black,bookmarksopen=true]{hyperref} % links em preto
\usepackage[pdftex,plainpages=false,pdfpagelabels,pagebackref,colorlinks=true,citecolor=DarkGreen,linkcolor=NavyBlue,urlcolor=DarkRed,filecolor=green,bookmarksopen=true]{hyperref} % links coloridos
\usepackage[all]{hypcap}                    % soluciona o problema com o hyperref e capitulos
\usepackage[square,sort,nonamebreak]{natbib} % citação bibliográfica textual(plainnat-ime.bst)
\bibpunct{[}{]}{;}{a}{\hspace{-0.7ex},}{,} % estilo de citação. Veja alguns exemplos em http://merkel.zoneo.net/Latex/natbib.php

\fontsize{60}{62}\usefont{OT1}{cmr}{m}{n}{\selectfont}

%%%%%----MIO
\usepackage{float}
\usepackage{array,multirow}
%\usepackage{enumerate}
%\usepackage{tikz}
\usepackage{subfig}
\usepackage{amsmath}
\usepackage{amsthm}
\usepackage{amsfonts}
\usepackage{amssymb}
%\theoremstyle{plain}
%Teoremas, definiciones y ejemplos
\newtheorem{thm}{Teorema}[section] % reset theorem numbering for each chapter
\theoremstyle{definition}
\newtheorem{defn}[thm]{Definición} % definition numbers are dependent on theorem numbers
\newtheorem{exmp}[thm]{Example} % same for example numbers
%\renewcommand{\figurename}{Figura.}
%\renewcommand{\tablename}{Tabla.}
\addto\captionsenglish{\renewcommand{\figurename}{Figura.}}
\addto\captionsenglish{\renewcommand{\tablename}{Tabla.}}
\usepackage{wrapfig}
%%%%%---MIO
%%% tabla
%\usepackage[table]{xcolor}
%\usepackage[margin=1in]{geometry}
\usepackage{tabularx}
\usepackage{enumitem}
\setlist{nolistsep}
\definecolor{green}{HTML}{66FF66}
\definecolor{myGreen}{HTML}{009900}
\usepackage{appendix}

%\renewcommand{\familydefault}{\sfdefault}
\renewcommand{\arraystretch}{1.5}

\renewcommand\spanishtablename{Tabla}
\usepackage{csvsimple}
\usepackage{pdfpages}

\titleformat{\chapter}[display]
{\normalfont\huge\bfseries}{\chaptertitlename\ \thechapter}{2pt}{\Large\MakeUppercase}
  
\titleformat{\section}{\normalfont\scshape\bfseries}{\textbf{\thesection}}{1em}{\MakeUppercase}
\titleformat{\subsection}{\normalfont\bfseries}{\textbf{\thesubsection}}{1em}{}
%\allsectionsfont{\mdseries\scshape}


\usepackage{hyperref}

%%%
% ---------------------------------------------------------------------------- %
% Cabeçalhos similares ao TAOCP de Donald E. Knuth
\usepackage{fancyhdr}
\pagestyle{fancy}
\fancyhf{}
\renewcommand{\chaptermark}[1]{\markboth{\MakeUppercase{#1}}{}}
\renewcommand{\sectionmark}[1]{\markright{\MakeUppercase{#1}}{}}
%\renewcommand{\chaptermark}[1]{\markboth{{#1}}{}}
%\renewcommand{\sectionmark}[1]{\markright{{#1}}{}}
\renewcommand{\headrulewidth}{0pt}

%\usepackage{caption}
% ---------------------------------------------------------------------------- %
\graphicspath{{./figuras/}}             % caminho das figuras (recomendável)
\frenchspacing                          % arruma o espaço: id est (i.e.) e exempli gratia (e.g.) 
\urlstyle{same}                         % URL com o mesmo estilo do texto e não mono-spaced
\makeindex                              % para o índice remissivo
\raggedbottom                           % para não permitir espaços extra no texto
\fontsize{60}{62}\usefont{OT1}{cmr}{m}{n}{\selectfont}
\cleardoublepage
\normalsize
%%Comment Out
\newcommand{\commentOut}[1]{}
% ---------------------------------------------------------------------------- %
% Opções de listing usados para o código fonte
% Ref: http://en.wikibooks.org/wiki/LaTeX/Packages/Listings
\lstset{ %
language=Java,                  % choose the language of the code
basicstyle=\footnotesize,       % the size of the fonts that are used for the code
numbers=left,                   % where to put the line-numbers
numberstyle=\footnotesize,      % the size of the fonts that are used for the line-numbers
stepnumber=1,                   % the step between two line-numbers. If it's 1 each line will be numbered
numbersep=5pt,                  % how far the line-numbers are from the code
showspaces=false,               % show spaces adding particular underscores
showstringspaces=false,         % underline spaces within strings
showtabs=false,                 % show tabs within strings adding particular underscores
frame=single,	                % adds a frame around the code
framerule=0.6pt,
tabsize=2,	                    % sets default tabsize to 2 spaces
captionpos=b,                   % sets the caption-position to bottom
breaklines=true,                % sets automatic line breaking
breakatwhitespace=false,        % sets if automatic breaks should only happen at whitespace
escapeinside={\%*}{*)},         % if you want to add a comment within your code
backgroundcolor=\color[rgb]{1.0,1.0,1.0}, % choose the background color.
rulecolor=\color[rgb]{0.8,0.8,0.8},
extendedchars=true,
xleftmargin=10pt,
xrightmargin=10pt,
framexleftmargin=10pt,
framexrightmargin=10pt
}

\lstdefinestyle{customc}{
  escapeinside={\%*}{*)},
  captionpos = b,
  belowcaptionskip=1\belowcaptionskip,
  breaklines=true,
  numbers=left,
  numbersep=5pt,
  numberstyle=\tiny\color{gray},
  language=C++,
  showstringspaces=false,
  basicstyle=\footnotesize\ttfamily,
  morekeywords={*,para,hasta,inicio,fin,repetir,veces},
  keywordstyle=\bfseries\color{green!40!black},
  commentstyle=\itshape\color{purple!40!black},
  identifierstyle=\color{blue},
  stringstyle=\color{orange},
  emph = {Union}, emphstyle = \color{purple},
  emph = {[2]Find}, emphstyle = {[2]\color{purple}}
}
\renewcommand*\lstlistingname{Algoritmo}
%----------------------------%
\hyphenation{per-so-nal res-pec-to}


% ---------------------------------------------------------------------------- %
% Corpo do texto
\parindent=2.5em
\begin{document}
\frontmatter 
% cabeçalho para as páginas das seções anteriores ao capítulo 1 (frontmatter)
\fancyhead[RO]{{\footnotesize\rightmark}\hspace{2em}\thepage}
\setcounter{tocdepth}{4}
\setcounter{secnumdepth}{4}
\fancyhead[LE]{\thepage\hspace{2em}\footnotesize{\leftmark}}
\fancyhead[RE,LO]{}
\fancyhead[RO]{{\footnotesize\rightmark}\hspace{2em}\thepage}
\onehalfspacing  % espaçamento

% ---------------------------------------------------------------------------- %
% CAPA
% Nota: O título para as dissertações/teses do IME-USP devem caber em um 
% orifício de 10,7cm de largura x 6,0cm de altura que há na capa fornecida pela SPG.
\thispagestyle{empty}

\begin{center}

\textsc{\large Universidad Nacional de San Antonio Abad del Cusco}\\\vspace*{0.04in}
FACULTAD DE CIENCAS QUÍMICAS, FÍSICAS Y MATEMÁTICAS\\
\vspace*{0.04in}
CARRERA PROFESIONAL DE INGENIERÍA INFORMÁTICA Y DE SISTEMAS \\

\captionsetup[figure]{labelformat=empty}
\begin{figure}[htb]
\begin{center}
\includegraphics[width=6cm]{img/UNSAAC}
\caption[]{}
\end{center}
\end{figure}

\newcommand{\topline}{
  \rule{164mm}{2mm}
  \vspace*{-0.23in}
  \hrule  
}
\newcommand{\downline}{
  \hrule  
  \vspace*{0.02in}
  \rule{164mm}{2mm}
}

\vspace*{-0.6in}
\textbf{TESIS}\\
\topline
\vspace*{0.1in}
\begin{large}
\textbf{``NUEVO ENFOQUE PARA LA SEGMENTACIÓN DE TEXTO EN FOTOGRAFÍAS QUE CONTENGAN SEÑALES INFORMATIVAS - DIRECCIONALES DE LAS CALLES DE CUSCO BASADO EN K-MEANS Y DISJOINT SETS''} \\
\end{large}
\vspace*{0.08in}
\downline
\vspace*{0.25in}

\begin{minipage}{\linewidth}
  \large
  \centering    
  \begin{minipage}{0.45\linewidth}
  \end{minipage}
  \hspace{0.28\linewidth}
  \begin{minipage}{0.7\linewidth}
    \begin{normalsize}
    Para optar al título profesional de:
    \vspace*{-0.1in}
    \\\textbf{INGENIERO INFORMÁTICO Y DE SISTEMAS}\\
    Presentado por: \vspace*{-0.1in}
    \\\textbf{Br. BERTHIN SANDINO TORRES CALLAÑAUPA}\\
    Asesor: \vspace*{-0.1in}
    \\\textbf{M.C.S LAURO ENCISO RODAS}\\
    Co-asesores: \vspace*{-0.1in}
    \\\textbf{M. Eng. E. GLADYS CUTIPA ARAPA} \vspace*{-0.1in}
    \\\textbf{Ph. D. ROSA ENSICO BACA}
    \end{normalsize}
  \end{minipage}
\end{minipage}

\vspace*{0.5in}
\textbf{CUSCO - PERÚ} \\
\textbf{2014}
\end{center}

\pagenumbering{roman}     % começamos a numerar 

% ---------------------------------------------------------------------------- %
% Agradecimentos:
% Se o candidato não quer fazer agradecimentos, deve simplesmente eliminar esta página 
\chapter*{Dedicatoria}
%A mi padre Berthin, mi madre Frida, y mi hermana Ilse.
En memoria de mi mamá Vicentina. Dedicado a mis padres Frida y Berthin.
%\chapter*{Agradecimientos}
%A mis asesores.
% ---------------------------------------------------------------------------- %

\chapter*{Resumen}

Dado que la segmentación de texto es uno de los pasos más importantes dentro 
del proceso de reconocimiento de texto en imágenes, este trabajo esta enfocado
en el texto localizado en los letreros informativos - direccionales ubicados 
en la ciudad de Cusco, Perú. Se identificaron problemas relacionados a la 
naturaleza de la imagen (exposición, calidad, y distorción) y problemas con 
respecto al estado físico de los letreros (erosión, daño, dimensiones, 
inclinación, suciedad). Con la finalidad de afrontar estos problemas, se 
propone un nuevo enfoque basado en una técnica no supervisada de aprendizaje 
(\textit{K-means clustering}) junto a la estructura de datos \textit{Disjoint 
sets} con la finalidad de reducir la complejidad computacional del 
\textit{K-means}.
\\

\noindent \textbf{Palabras clave:} Segmentación de texto, Stroke Witdh 
Transform, K-means, Disjoint-sets, Union-Find, Métodos para binarizar imágenes
(Niblack, Otsu, Bernsern, Triangle), Filtro Media, Filtro Mediana.

% ---------------------------------------------------------------------------- %
% Abstract
\chapter*{Abstract}
%\noindent\textbf{...}

Since text segmentation is one of the most important steps in text recognition
process, this work is focused on text of hand crafted ceramic tile street 
signs located throughout the City of Cusco, Peru. Problems that were 
identified involve the nature of the image (exposure, image quality, and 
distortion) and the physical state of the street sign (maintenance, damage, 
size, and leveling). In order to deal with these points, a novel approach is 
presented combining an unsupersived machine learning technique 
(\textit{K-means clustering}) used together with Disjoint sets data structure 
to reduce the computational complexity of \textit{K-means}.
\\

\noindent \textbf{Keywords:} Text Segmentation, Stroke Witdh Transform, 
K-means, Disjoint-sets, Union-Find, Thresholding methods (Niblack, Otsu, 
Bernsern, Triangle), Mean filter, Fast Median Filter, Local and global 
methods for Image Processing.

\tableofcontents    % imprime o sumário

% ---------------------------------------------------------------------------- %
% Listas de figuras e tabelas criadas automaticamente
\renewcommand\listtablename{ÍNDICE DE TABLAS}
\renewcommand\listfigurename{ÍNDICE DE FIGURAS}
\listoftables
\listoffigures

% ---------------------------------------------------------------------------- %
% Capítulos do trabalho
\mainmatter

% cabeçalho para as páginas de todos os capítulos
\fancyhead[RE,LO]{\thesection}

%\singlespacing              % espaçamento simples
\onehalfspacing            % espaçamento um e meio

\input cap-introducion
\part{ASPECTOS GENERALES}
\input cap-aspectosgenerales

\part{MARCO TEÓRICO}
\input cap-marcoteorico
\input cap-reconocimiento
\input cap-swt

\part{DESARROLLO DE LA ESTRUCTURA DEL ENFOQUE PROPUESTO}
\input cap-aspectosenfoque  
\input cap-preprocesamiento
\input cap-segmentacion
\input cap-resultados

\input cap-conclusiones
\input cap-trabajosfuturos

% ---------------------------------------------------------------------------- %
% Bibliografia
\renewcommand\bibname{BIBLIOGRAFÍA}
\backmatter \singlespacing   % espaçamento simples
\bibliographystyle{ieeetr} % citação bibliográfica textual
\bibliography{bibliografia}  % associado ao arquivo: 'bibliografia.bib'

%-------------------------------------------
%\include{ape-conjuntos}      % associado ao arquivo: 'ape-conjuntos.tex'
\renewcommand{\appendixname}{Anexo}
\renewcommand{\appendixtocname}{ANEXOS}
\renewcommand{\appendixpagename}{ANEXOS}
\renewcommand{\chaptermark}[1]{\markboth{\MakeUppercase{\appendixname\ \thechapter}} {\MakeUppercase{#1}} }
\fancyhead[RE,LO]{}
\appendix
\clearpage
\addappheadtotoc
\appendixpage
\include{ape-results}
\include{ape-laptop}

\end{document}
