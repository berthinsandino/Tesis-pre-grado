\chapter{ASPECTOS DEL ENFOQUE}
\label{cap-aspectosenfoque}
%\section{Segmentación de texto}
\setcounter{secnumdepth}{0}
%\section{Introducción}

La segmentación de texto en imágenes implica separar los \textit{pixels} 
pertenecientes al texto (\textit{foreground}) de los \textit{pixels} que forman 
parte del \textit{background}, este proceso representa una parte crítica en cualquier
módulo de reconocimiento de texto.

Dado que elfoco central del trabajo es el problema de segmentación 
(dejando de lado la fase de reconocimiento de texto para trabajos futuros), en
la siguiente sección se presentará el enfoque propuesto, dejando el desarrollo
para los capítulos dos siguientes capítulos.

\setcounter{secnumdepth}{3}
\section{Enfoque propuesto}

El problema de segmentación será abordado en tres fases (Figura~\ref{fig:cap-reconocimiento:workflowenfoque}):
\begin{itemize}
	\item La primera fase (Pre-procesamiento) llevará a cabo procesos que filtren características
	que no nos sean útiles.
	\item La segunda fase (Segmentación de Texto) llevará a cabo el proceso de segmentación así como 
	localización eliminando aquellas regiones que no formen parte del texto.
	\item La tercera fase (Post-procesamiento) será la encarga de filtrar componentes falsos 
	positivos.
\end{itemize}

\begin{figure}[h!]
	\centering
	\includegraphics[height=15cm]{Cap:Reconocimiento/workflow_propuesta2}
	\caption{Enfoque propuesto}
	\label{fig:cap-reconocimiento:workflowenfoque}
\end{figure}

Durante la fase de segmentación de texto, la fotografía será sometida a una
serie de filtros con el objetivo de eliminar regiones cuya información no sea
relevante para nuestro objetivo. Dado que, el punto de interés es la
segmentación de texto ubicado dentro de los letreros informativos -
direccionales. Se considerarán características que ayuden a reducir las áreas 
de búsqueda. 

El objetivo del uso de los primeros filtros será el de eliminar los 
\textit{pixels} que sean poco probables de pertenecer al letrero, para lo cual
se hará uso de un filtro que deje pasar aquellos \textit{pixels} cuyos valores
de color estén ubicados dentro de un intérvalo pre-establecido. El resultado del
primer filtro será sometido a otros dos filtros con la finalidad de eliminar
regiones de la imagen que no aporten información al proceso de segmentación de
texto.

Luego, se formarán componentes conexas sobre la nueva imagen; para que, 
mediante el KM se formen 3 \textit{clusters}, dicho de otra forma, se genere 
una imagen con solo 3 colores (colorA, colorB y colorC).

Como el texto ubicado en los letreros originalmente presenta una distribución 
de color constante, podemos asumir que, por lo menos uno de los tres colores
formará parte del texto, es decir, se genera una imagen considerándo únicamente
un subconjunto de los 3 \textit{clusters} formados.

Finalmente, se trabaja con algunas heurísticas sobre las dos imágenes 
eliminando regiones que no sean candidatas a ser texto.
