\chapter{PRE-PROCESAMIENTO}
\label{cap-preprocesamiento}
\setcounter{secnumdepth}{0}
Antes de poder extraer algún tipo de información relevante de las imágenes
adquiridas, es necesario un pre-procesamiento básico de las mismas. El objetivo
de este paso es la eliminación de regiones dentro de la imagen cuya información
no es de interés para el presente trabajo, de esta forma se puede alterar el
problema original (con muchas restricciones) a otro con características que
puedan ser explotadas.

\setcounter{secnumdepth}{3}
\section{\textit{Color Filtering}}
Considerando que los letreros sobre los cuales se basa el presente trabajo
comparten tonalidades diferentes del color azul tanto para el marcado de la
señal como para el texto, se aprovecha esta característica para limitar las
regiones de búsqueda. Este filtrado será de gran utilidad debido a que las
fotografías consideradas dentro del \textit{training data set} poseen
dimensiones de tamaño considerable.

\subsection{Filtro \textit{Ad hoc}}
La finalidad del uso del presente filtro es dejar pasar aquellos
\textit{pixels} que se encuentren dentro de un rango de color pre-establecido.

Un total de 14 muestras fueron utilizadas durante el análisis para determinar
el mejor rango de color, tomándose en cuenta 2 modelos de color (RGB y HSV).
Como se observa en el histograma del modelo RGB
(Figura~\ref{Fig:Cap-segmentaciontexto:HistogramaRGB}), el posible rango sería
$[0,~250]$ para R, $[0,~200]$ para G y $[0,~150]$ para B. La
Figura~\ref{Fig:Cap-segmentaciontexto:coloresDiferencia} es un ejemplo donde el
uso del rango de valores permitiría que se tomen en consideración colores ajenos
al letrero, mientras que en la Figura~\ref{Fig:Cap-segmentaciontexto:errorRGB}
se muestran los histogramas para dos fotografías del \textit{data set} donde el
rango de variación de los colores del letrero no permiten definir un rango
general.

\begin{figure}[h!]
	%\centering
	\includegraphics[width=15cm]{/Cap:SegmentacionTexto/histogramaRGB}
	\caption{Histograma RGB resultado del análisis de las 14 muestras.}
	\label{Fig:Cap-segmentaciontexto:HistogramaRGB}
\end{figure}

\begin{figure}[h!]
	\centering
	\includegraphics[width=15cm]{/Cap:SegmentacionTexto/coloresDiferencia}
	\caption{Cinco muestras aceptadas siguiendo la sugerencia del Histograma RGB.}
	\label{Fig:Cap-segmentaciontexto:coloresDiferencia}
\end{figure}

\begin{figure}[h!]
	\centering
	%\subfloat[]{\includegraphics[width=7cm]{/home/berthin/Pictures/Letreros/03.jpg}\label{Fig:Cap-segmentaciontexto:errorRGB.a}} { }	
	%\subfloat[]{\includegraphics[width=7cm]{/home/berthin/Pictures/Letreros/32.jpg}\label{Fig:Cap-segmentaciontexto:errorRGB.b}} { }\\
	\subfloat[]{\includegraphics[width=7cm]{/Cap:SegmentacionTexto/histogramaRGB_Letreros_03.png}\label{Fig:Cap-segmentaciontexto:errorRGB.a}} { }
  \subfloat[]{\includegraphics[width=7cm]{/Cap:SegmentacionTexto/histogramaRGB_Letreros_32}\label{Fig:Cap-segmentaciontexto:errorRGB.b}} \\
	\subfloat[]{\includegraphics[width=7cm]{/Cap:SegmentacionTexto/histogramaRGB_Letreros_03}\label{Fig:Cap-segmentaciontexto:errorRGB.c}} { }
  \subfloat[]{\includegraphics[width=7cm]{/Cap:SegmentacionTexto/histogramaRGB_Letreros_32}\label{Fig:Cap-segmentaciontexto:errorRGB.d}} \\
	\caption{(c y d) Histogramas RGB de los colores del texto ubicado en los letreros (a y b) respectivamente.}
	\label{Fig:Cap-segmentaciontexto:errorRGB}
\end{figure}

Como solución al problema encontrado con el modelo RGB, se consideraron otros
modelos de color (HSV y YCbCr). Si se observan los valores HSV en la
Figura~\ref{Fig:Cap-segmentaciontexto:coloresDiferencia} se podrá notar que la
variación yace principalmente en el valor H. El resultado del análisis mostrado
en la Figura~\ref{Fig:Cap-segmentaciontexto:HistogramaHSV} permite determinar un
rango de valores ($[180,~360]$ para H, $[0,~100]$ para S y $[0,~100]$ para V)
que al ser sometido a una serie de pruebas, los resultados obtenidos son
aceptables para el objetivo del problema de filtrado que ayudará a eliminar
partes de imágenes que no son de nuestro interés.

\begin{figure}[h!]
	\includegraphics[width=15cm]{/Cap:SegmentacionTexto/histogramaHSV}
	\caption{Histograma HSV resultado del análisis de las 14 muestras.}
	\label{Fig:Cap-segmentaciontexto:HistogramaHSV}
\end{figure}

\begin{figure}[h!]
	\centering
  \subfloat[]{\includegraphics[width=7cm]{/Cap:SegmentacionTexto/filtroAdHoc_Letreros_03}\label{Fig:Cap-segmentaciontexto:filtroAdHocHSV.a}} { }
  \subfloat[]{\includegraphics[width=7cm]{/Cap:SegmentacionTexto/filtroAdHoc_Letreros_32}\label{Fig:Cap-segmentaciontexto:filtroAdHocHSV.b}} \\
	\caption{Resultado del filtro usando los parámetros definidos para el modelo
	HSV de las imágenes (a y b) de la 
	figura~\ref{Fig:Cap-segmentaciontexto:errorRGB} respectivamente.}
	\label{Fig:Cap-segmentaciontexto:filtroAdHocHSV}
\end{figure}

\subsection{Filtro usando NNs}
Las redes neuronales permiten desarrollar un filtro que use información extra
para determinar o clasificar cada elemento de una imagen en dos categorías o
grupos (la \textbf{clase 1} representará al \textit{background} y la
\textbf{clase 2} el \textit{foreground}). Mediante este enfoque, se construyó un
\textit{Feed-forward network} de 2 capas
(Figura~\ref{Fig:Cap-segmentaciontexto:diagramaNN}). La primera capa
(\textit{input layer}), con 75 \textit{features} será la encargada de brindar
las entradas a la NN. La segunda capa (\textit{hidden layer}), cuenta con 10
neuronas; y la última capa (\textit{output layer}), con 2 neuronas. La última
parte de la NN estará conformada por 2 unidades de salida que representan las
categorías de clasificación.

La \textit{hidden layer} y la \textit{output layer} cuentan con unidades 
adicionales de entrada llamadas \textit{bias units}.

Los \textit{features} considerados como entrada para la NN fueron obtenidos 
haciendo uso de la información de los \textit{pixels} vecinos. Si $P(x,y)$
representaría un \textit{pixel} sobre el cual se desea trabajar, mediante una
ventana de $5 \times 5$ (con $P$ como punto central) capturamos 25
\textit{pixels} cada uno con 3 tipos de información (valores RGB). Todo ello,
formará parte de la entrada para la NN. 

\begin{figure}[h!]
	\centering
  \subfloat[]{\includegraphics[height=3cm]{/Cap:SegmentacionTexto/inputNN}\label{Fig:Cap-segmentaciontexto:inputNN}} { }
  \subfloat[]{\includegraphics[height=3cm]{/Cap:SegmentacionTexto/filtroNN_Diagrama}\label{Fig:Cap-segmentaciontexto:diagramaNN}} \\
	\caption{Ventana de $5 \times 5$ considerada para el pixel marcado con color
	rojo. Diagrama de la NN.}
	\label{Fig:Cap-segmentaciontexto:input&diagramaNN}
\end{figure}

A partir de un conjunto de 7 imágenes 
(Figura~\ref{Fig:Cap-segmentaciontexto:dataNN}) que contengan letreros
informativos se construyó una \textit{data set} de tamaño $198555 \times 75$.

\begin{figure}[h!]
	\centering
  \subfloat[]{\includegraphics[height=1.5cm]{/Cap:SegmentacionTexto/NN_img01}\label{Fig:Cap-segmentaciontexto:diagramaNN.a}} { }
  \subfloat[]{\includegraphics[height=1.5cm]{/Cap:SegmentacionTexto/NN_img02}\label{Fig:Cap-segmentaciontexto:diagramaNN.b}} { }
  \subfloat[]{\includegraphics[height=2cm]{/Cap:SegmentacionTexto/NN_img03}\label{Fig:Cap-segmentaciontexto:diagramaNN.c}} { }
  \subfloat[]{\includegraphics[height=2cm]{/Cap:SegmentacionTexto/NN_img04}\label{Fig:Cap-segmentaciontexto:diagramaNN.d}} \\
  \subfloat[]{\includegraphics[height=1cm]{/Cap:SegmentacionTexto/NN_img05}\label{Fig:Cap-segmentaciontexto:diagramaNN.e}} { }
  \subfloat[]{\includegraphics[height=1cm]{/Cap:SegmentacionTexto/NN_img07}\label{Fig:Cap-segmentaciontexto:diagramaNN.f}} { }
  \subfloat[]{\includegraphics[height=1cm]{/Cap:SegmentacionTexto/NN_img06}\label{Fig:Cap-segmentaciontexto:diagramaNN.g}} \\
	\caption{Imágenes usadas para crear el \textit{data set} para la NN.}
	\label{Fig:Cap-segmentaciontexto:dataNN}
\end{figure}

Aleatóriamente, se derivó el $70 \%$ para la fase de \textit{training}, 
$15 \%$ para \textit{testing} y $15 \%$ para la fase de \textit{validation}. El
algoritmo \textit{scaled conjugate gradient backpropagation}
\cite{Moller:1993:ScaledGD} fue usado durante el entrenamiento con el objetivo
de actualizar los pesos de las conexiones entre neuronas así como los valores
del las \textit{bias units}. Un resumen de fases de \textit{training, testing} y
\textit{validation} se muestra en la
Figura~\ref{Fig:Cap-segmentaciontexto:plots1.a} mediante las matrices de
confusión (\textit{confusion matrices}) donde el color verde está relacionado
con las respuestas correctas, el color rojo a las respuestas incorrectas y el
color azul con el resultado global, y en la
Figura~\ref{Fig:Cap-segmentaciontexto:plots1.b} se muestra la matriz de
confusión para la Figura~\ref{Fig:Cap-segmentaciontexto:diagramaNN} cuyo
resultado final se muestra en la
Figura~\ref{Fig:Cap-segmentaciontexto:resultadoNN.a}.

\begin{figure}[h!]
	\centering
  \subfloat[]{\includegraphics[width=8cm]{/Cap:SegmentacionTexto/plotconfusion}\label{Fig:Cap-segmentaciontexto:plots1.a}} { }
  \subfloat[]{\includegraphics[width=8cm]{/Cap:SegmentacionTexto/plotconfusion_test}\label{Fig:Cap-segmentaciontexto:plots1.b}} \\
  \caption[\textit{Plot matrices confusion}]{\textit{Plot matrices confusion}.
  (a) Resumen de las fases de entrenamiento. (b) Resultado de un \textit{test}
adicional.}
	\label{Fig:Cap-segmentaciontexto:plots1}
\end{figure}

\begin{figure}[h!]
	\centering
  \includegraphics[width=8cm]{/Cap:SegmentacionTexto/confusionMatrixNN1} { }
	\includegraphics[width=8cm]{/Cap:SegmentacionTexto/confusionMatrixNN2}
	\caption{\textit{Confusion matrices} de las dos NNs.}
	\label{Fig:Cap-segmentaciontexto:confusionMatrixNNs}
\end{figure}

Por otra parte, se entrenó otra NN con las mismas características, pero a 
diferencia de la presentada anteriormente, esta cuenta con 12 neuronas en la
\textit{hidden layer} y con un \textit{data set} de mayor tamaño ($247099 \times
75)$. Comparando resultados entre las dos NNs (Figura
~\ref{Fig:Cap-segmentaciontexto:confusionMatrixNNs}), la primera presenta un
mejor desempeño que la segunda debido a que se necesitaría un \textit{data set}
de mayor tamaño y por ende un equipo de cómputo superior para que se agilicen
los cálculos matemáticos involucrados en la fase de entrenamiento. La
Figura~\ref{Fig:Cap-segmentaciontexto:resultadoNN} muestra los resultados
obtenidos mediante el filtro NN sobre las mismas imágenes usadas en la
Figura~\ref{Fig:Cap-segmentaciontexto:filtroAdHocHSV}.

\begin{figure}[h!]
	\centering
  \subfloat[]{\includegraphics[width=7cm]{/Cap:SegmentacionTexto/filtroNN_Letreros_03}\label{Fig:Cap-segmentaciontexto:resultadoNN.a}} { }
  \subfloat[]{\includegraphics[width=7cm]{/Cap:SegmentacionTexto/filtroNN_Letreros_32}\label{Fig:Cap-segmentaciontexto:resultadoNN.b}} \\
	\caption{Resultado del filtro NN para las imágenes de las figuras~\ref{Fig:Cap-segmentaciontexto:errorRGB.a} y \ref{Fig:Cap-segmentaciontexto:errorRGB.b} respectivamente.}
	\label{Fig:Cap-segmentaciontexto:resultadoNN}
\end{figure}

\section{\textit{Smoothing Spatial Filters}}
Las imágenes que resulten del \textit{color filtering} por lo general tienden 
a poseer pequeñas regiones las cuales, en su gran mayoría, no son relevantes
para nuestro fin, por ello que nace la necesidad de eliminar o suavizar ese
ruido.

\subsection{\textit{Mean filter}}
La idea de usar el \textit{mean filter} es reemplazar los valores de los 
\textit{pixels} de la imagen con la media arimética de sus vecinos (incluyendo
al mismo \textit{pixel}). La
Figura~\ref{Fig:Cap-segmentaciontexto:filterMean-Median.c} fue generada usando
el kernel expresado en \ref{Equ:Cap-segmentaciontexto:Kernel-mean}.
\begin{equation}
K =  \left( \begin{array}{ccc}
1/3 & 1/3 & 1/3 \\
1/3 & 1/3 & 1/3 \\
1/3 & 1/3 & 1/3 \end{array} \right)
\label{Equ:Cap-segmentaciontexto:Kernel-mean}
\end{equation}

\subsection{\textit{Median filter}}
Filtro cuyo objetivo es reducir el ruido de una imagen reemplazando el valor 
de un \textit{pixel} $p$ con aquel que se encuentre en la mediana de los valores
de sus vecinos (también se toma en consideración el valor del \textit{pixel}
$p$). 

Siendo $G^{n,m}$ una imagen en escala de grises, el \textit{median filter}
usado está para generar la
Figura~\ref{Fig:Cap-segmentaciontexto:filterMean-Median.d} está definido
mediante la ecuación~\ref{Equ:Cap-segmentaciontexto:median}.

\begin{equation}
	G' = Md \left( N_8(p)~ \bigcup ~p \right), ~\forall p \in G
	\label{Equ:Cap-segmentaciontexto:median}
\end{equation}

\begin{figure}[h!]
	\centering
  \subfloat[]{\includegraphics[width=8cm]{/Cap:SegmentacionTexto/filtroAdHoc_RGB_FotosCalles_002}\label{Fig:Cap-segmentaciontexto:filterMean-Median.a}} { }
  \subfloat[]{\includegraphics[width=8cm]{/Cap:SegmentacionTexto/filtroAdHoc_Gray_FotosCalles_002}\label{Fig:Cap-segmentaciontexto:filterMean-Median.b}} \\
  \subfloat[]{\includegraphics[width=8cm]{/Cap:SegmentacionTexto/filtroMean_FotosCalles_002}\label{Fig:Cap-segmentaciontexto:filterMean-Median.c}} { }
  \subfloat[]{\includegraphics[width=8cm]{/Cap:SegmentacionTexto/filtroMedian_FotosCalles_002}\label{Fig:Cap-segmentaciontexto:filterMean-Median.d}} \\
  \caption[Filtros \textit{mean} y \textit{median}]{Filtros \textit{mean} y 
  \textit{median}. (a) Imagen original. (b) Escala de grises. (c) \textit{Mean
filter}. (d) \textit{Median filter}}
	\label{Fig:Cap-segmentaciontexto:filterMean-Median}
\end{figure}

\section{\textit{Thresholding}}
Las imágenes en escala de grises que fueron sometidas a la fase de 
\textit{color filtering} así como también a la fase de \textit{smoothing}
(Figuras~\ref{Fig:Cap-segmentaciontexto:filterMean-Median.c} y
\ref{Fig:Cap-segmentaciontexto:filterMean-Median.d}) son binarizadas con el
objetivo de separar el \textit{foreground} del \textit{background}.

\subsection{\textit{Otsu's method}}
Este método, que forma parte de las técnicas de \textit{Clustering-based 
thresholding}, es uno de los más citados en las referencias bibliográficas. La
Figura~\ref{Fig:Cap-segmentaciontexto:thresholdingA.a} muestra el resultado de
su aplicación sobre imagen de la
Figura~\ref{Fig:Cap-segmentaciontexto:thresholdingA.b}.

\subsection{\textit{Niblack's method}}
A diferencia del método \textit{Otsu}, el método \textit{Niblack} usa valores
locales de una ventana de tamaño pre-determinado. La
Figura~\ref{Fig:Cap-segmentaciontexto:thresholdingA.b} es un ejemplo del
resultado obtenido del método usando una ventana de tamaño $50 \times 50$ y con
parámetros $k$ y $c$ iguales a $-0.2$ y $10$ respectivamente, estos los valores
fueron determinados mediante el método de prueba y error.

\begin{figure}[h!]
	\centering
	\setlength{\fboxsep}{0pt}
  \subfloat[]{\fbox{\includegraphics[width=8cm]{/Cap:SegmentacionTexto/otsu_FotosCalles:002.jpg}\label{Fig:Cap-segmentaciontexto:thresholdingA.a}}} { }
  \subfloat[]{\fbox{\includegraphics[width=8cm]{/Cap:SegmentacionTexto/niblack_FotosCalles:002.jpg}\label{Fig:Cap-segmentaciontexto:thresholdingA.b}}} \\
  \caption[\textit{Otsu and Niblack thresholding}]{\textit{Otsu and Niblack 
  thresholding}. (a) \textit{Otsu's method}. (b) \textit{Niblack's method}.}
	\label{Fig:Cap-segmentaciontexto:thresholdingA}
\end{figure}

\subsection{\textit{Bernsen's method}}
\textit{Bernsen's method} \cite{Bernsen:1986:DTofGI}, es un método de 
contraste local que trabaja sobre los valores mínimo y máximo de una (tamaño
sugerido $31 \times 31$). Si el contraste local es mayor o igual al
\textit{contrast threshold} (cuyo valor por defecto es $15$), el
\textit{threshold} tomará el valor del promedio local (la media entre el máximo
y mínimo valor de los valores de la ventana). Si el \textit{local contrast} es
menor que el \textit{contrast threshold}, el \textit{pixel} será establecido
como \textit{background} o \textit{foreground} dependiendo del valor del
promedio local.

\subsection{\textit{Triangle's method}}
\textit{Triangle's method} \cite{Zack:1977:AMofSCRF}, es un método geométrico 
basado en la construcción de una línea entre los valores mínimo y máximo
expuestos en el histograma para luego iterar entre ese rango y buscar la máxima
distancia entre la línea y el valor del histograma
(Figura~\ref{Fig:Cap-segmentaciontexto:triangle}). Aquel valor del histograma
cuya distancia hacia la línea sea máximo determinará el valor del
\textit{threshold}.
\begin{figure}[h]
	\centering
	\includegraphics[width=8cm]{/Cap:SegmentacionTexto/triangles_method.png}
  \caption[\textit{Triangle's Thresholding}.]{\textit{Triangle's 
  Thresholding}}\tiny{Fuente:
\url{http://www.mif.vu.lt/atpazinimas/dip/FIP/fip-Segmenta.html}}
	\label{Fig:Cap-segmentaciontexto:triangle}
\end{figure}

\begin{figure}[h]
	\centering
	\setlength{\fboxsep}{0pt}
  \subfloat[]{\fbox{\includegraphics[width=8cm]{/Cap:SegmentacionTexto/Bernsen_FotosCalles_002}\label{Fig:Cap-segmentaciontexto:thresholdingB.a}}} { }
  \subfloat[]{\fbox{\includegraphics[width=8cm]{/Cap:SegmentacionTexto/Triangle_FotosCalles_002}\label{Fig:Cap-segmentaciontexto:thresholdingB.b}}} \\
  \caption[\textit{Bernsen and Triangle thresholding}]{\textit{Bernsen and 
  Triangle thresholding}. (a) \textit{Bernsen's method}. (b) \textit{Triangle's
method}.}
	\label{Fig:Cap-segmentaciontexto:thresholdingB}
\end{figure}

\section{\textit{Merging results from filters}}
Una vez obtenida una imagen \textit{black and white}, mediante el operador 
binario \textit{AND} se procede a unir la imagen que resultó del
\textit{Smoothing Spatial Filter} con el inverso de la imagen obtenida haciendo
uso del \textit{Thresholding}
(Figura~\ref{Fig:Cap-segmentaciontexto:resultado-preprocesing}).

\begin{figure}[h]
	\centering
%	\setlength{\fboxsep}{0pt}
  \subfloat[]{\fbox{\includegraphics[width=8cm]{/Cap:SegmentacionTexto/final_Otsu_FotosCalles_002}\label{Fig:Cap-segmentaciontexto:resultado-preprocesing.a}}} { }
  \subfloat[]{\fbox{\includegraphics[width=8cm]{/Cap:SegmentacionTexto/final_Niblack_FotosCalles_002}\label{Fig:Cap-segmentaciontexto:resultado-preprocesing.b}}} \\
  \caption[Resultado de la fase de pre-procesamiento]{Resultado de la fase de 
  pre-procesamiento obtenido de la
Figura~\ref{Fig:Cap-segmentaciontexto:filterMean-Median.d} (\textit{median
filter}) con los resultados de: (a) \textit{Otsu's thresholding}. (b)
\textit{Niblack's thresholding}.}
	\label{Fig:Cap-segmentaciontexto:resultado-preprocesing}
\end{figure}
