\chapter{STROKE WIDTH TRANSFORM}
\label{cap-swt}
El 2010, Esphtein \textit{et al.} \cite{Epshtein:SWT:2010}, investigadores de 
\textit{Microsoft}, presentaron un novedoso \textit{image operator} durante la
\textit{23rd IEEE Conference on Computer Vision and Pattern Recognition} en San
Francisco\footnote{\url{http://videolectures.net/cvpr2010_epshtein_dtns/}} con
el objetivo de buscar el valor del \textit{stroke width} para cada
\textit{pixel} de una imagen. Para fines demostrativos, su operador fue usado
para la detección de texto en imágenes con un \textit{background} complejo. 

El algoritmo propuesto fue sometido a los \textit{data sets} de las 
competencias de localización de texto ICDAR 2003 y ICDAR 2005\cite{ICDAR:web}
debido a que los resultados obtenidos durante las competencias demostraron que
aún hay falencias que deben de ser estudiadas. Como resultado de los
experimentos, su propuesta presentó un mejor rendimiento que otros algoritmos 
y en un menor tiempo de ejecución.

Dentro del proceso de localización, también se lleva a cabo la tarea de 
segmentación de texto, y esta es la razón principal por la cual se tomará este
enfoque para comparar los resultados con los obtenidos por el enfoque propuesto.

El esquema usado por en el SWT para la segmentación de texto se describe el en 
Algoritmo~\ref{alg:swt}.
\lstinputlisting[caption=SWT, style=customc, label=alg:swt]{pseudocodigo/codigo4.cod}

\clearpage
\begin{figure}
	\setlength{\fboxsep}{0pt}
	\subfloat[]{\includegraphics[width=8cm]{Cap:SWT/001}\label{Fig:cap-swt:paso01.a}}{ }	
  \subfloat[]{\fbox{\includegraphics[width=8cm]{Cap:SWT/canny001}\label{Fig:cap-swt:paso01.b}}} \\
  \subfloat[]{\includegraphics[width=8cm]{Cap:SWT/001_gray}\label{Fig:cap-swt:paso01.c}} { }
	\subfloat[]{\includegraphics[width=8cm]{Cap:SWT/001_smooth}\label{Fig:cap-swt:paso01.d}}
	\caption[SWT - I]{SWT - I. (a)Imagen original. (b)\textit{Canny Edge 
	Detector}. (c)\textit{Grayscale}. (d)\textit{Smooth gaussian filter}.}
	\label{Fig:cap-swt:paso01}
\end{figure}

\begin{figure}
	\subfloat[]{\includegraphics[width=9cm]{Cap:SWT/Jx}\label{Fig:cap-swt:paso02.a}} 
  \subfloat[]{\includegraphics[width=9cm]{Cap:SWT/Jy}\label{Fig:cap-swt:paso02.b}} 
  \caption[SWT - II]{Gradientes $X$ y $Y$ obtenidos mediante el filtro 
  \textit{Sobel Edge Detector}. (a)\textit{Gradient X}. (b)\textit{Gradient Y}}
  \label{Fig:cap-swt:paso02}
\end{figure}

\begin{figure}
	\subfloat[]{\includegraphics[width=15cm]{Cap:SWT/swt}\label{Fig:cap-swt:paso03.a}}\\
  \subfloat[]{\includegraphics[width=15cm]{Cap:SWT/swtx2}\label{Fig:cap-swt:paso03.b}}
  \caption[SWT - III]{Vector gradiente para cada \textit{pixel}. 
  (a)\textit{Gradient vectors}. (b)\textit{Gradient vectors (zoom)}}
  \label{Fig:cap-swt:paso03}
\end{figure}

\begin{figure}
	\setlength{\fboxsep}{0pt}
	\subfloat[]{\fbox{\includegraphics[width=15cm]{Cap:SWT/SWT001}\label{Fig:cap-swt:paso04.a}}}\\
  \subfloat[]{\fbox{\includegraphics[width=15cm]{Cap:SWT/components001}\label{Fig:cap-swt:paso04.b}}}
  \caption[SWT - IV]{SWT - IV. (a)SWT. (b)\textit{Connected components}.}
  \label{Fig:cap-swt:paso04}  
\end{figure}

\begin{figure}[h!]
	\setlength{\fboxsep}{0pt}
	\fbox{\includegraphics[width=15cm]{Cap:SWT/output_001}}
  \caption[SWT - V]{Resultado: \textit{Valid chains}.}
  \label{Fig:cap-swt:paso05}	
\end{figure}
