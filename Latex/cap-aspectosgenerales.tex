%% ------------------------------------------------------------------------- %%
\chapter{ASPECTOS GENERALES}
\label{cap:aspectosGenerales}
%% ------------------------------------------------------------------------- %%
\setcounter{secnumdepth}{0}

\setcounter{secnumdepth}{3}
\section{Planteamiento del Problema}
\subsection{Descripción}

El reconocimiento de texto en imágenes\footnote{Las imágenes se pueden
clasificar en dos tipos considerando su origen. \textbf{Gráficos}, aquellas
imágenes obtenidas a partir de una aplicación de computadora. Y 
\textbf{fotografías}, imágenes capturadas mediante un dispositivo electrónico.},
durante los último años se fue transformando en uno de los temas más estudiados
dentro del área de Reconocimiento de Patrones, debido a que, el texto provee
información semántica para el entendimiento humano. {\color{red} referencia}

Con el avance tecnológico\footnote{Berman \citep{Berman:2003:GCM}, en el 2003,
realizó un estudio sobre la cantidad de información que se podría generar dentro
de los próximos 10 años. Considerando únicamente información de carácter
científico (obtenida de satelitales, radares, y escáneres médicos), se
produciría entre 100 \textit{Terabytes} ($10^{12}$ \textit{bytes}) a 10
\textit{Petabytes} ($10^{15}$ \textit{bytes}) por año. Por otra parte, en el
2013, Eric Schmidt CEO (\textit{Chief Executive Officer}) de Google, el más
grande indexador de Internet, estimó que el tamaño de información estaría
bordeando los 5 millones de \textit{Terabytes} \citep{McGuigan:2013:HBI}.} nace
la necesidad de digitalizar la información, lo cual ha llevado a desarrollar
diversas metodologías con desempeños que experimentan cambios favorables o
adversos según sea el tipo de problema que se enfrente. Del mismo modo, la
masificación del uso del Internet, ya sea mediante correos electrónicos o redes
sociales como medios para compartir información, ha logrado que actualmente
tengamos nuevos problemas para afrontar.

La segmentación de texto representa uno de los pasos más importantes en el
proceso de reconocimiento en imágenes, debido a que debe lidiar con los
problemas de tener imágenes con un \textit{background} complejo, especialmente
en estos días, cuando la tecnología nos provee de múltiples formas para la
captura de fotografías y con niveles de resolución que cada día mejoran de
manera vertiginosa.\footnote{En la actualidad se poseen cámaras fotográficas (no
profesionales) de 12, 14 y 16 Mpx (Megapíxeles) y se espera que en pocos años se
cuenten con cámaras de 25 Mpx, pero esto no se dentendrá allí, ya que según
estimaciones, el ojo humano cuenta con una resolución mayor a 500 Mpx. Muchas
imágenes digitales usan 24 \textit{bits} por cada \textit{pixel}, sin embargo,
se cuenta con cámaras que usan 36 \textit{bits} y si tomamos en cuenta la Ley de
Moore \citep{Moore:1965:CCIC}, en unos años tendremos cámaras que soporten 48 y
luego 96 \textit{bits} \citep{Myhrvold:2006:MLCPP}.}

En la ciudad de Cusco, uno de los lugares más turísticos del Perú, se tienen
distribuidos a lo largo de sus calles letreros informativos - direccionales con
una característica muy peculiar, están formados por baldosas de cerámica.

Luego de la influencia de la cultura Europea, muchos nombres de las calles de la
ciudad de Cusco fueron modificados dejándo de lado los nombres quechua
originales.\footnote{Los cuales almacenaban información relacionada al día a día
de los inkas, así como, información al respecto de los lugares.} Recién, a
partir de 1993 ---durante el gobierno edil del Dr. Daniel Estrada Perez---, como
parte de una campaña para rescatar la identidad local e historia de nuestra
cultura,  se oficializó la recuperación de los nombres antiguos de las calles de
la ciudad de Cusco. La Municipalidad Provincial de Cusco conjuntamente con la
Gerencia del Ministerio de Transportes y Comunicaciones, fueron colocando
letreros de señalización y/o información, muchos de ellos sobre baldosas de
cerámica.\footnote{Letreros que se rigen al ``Manual de Dispositivos de Control
del Tránsito Automotor en Calles y Carreteras'' (aprobado por Resolución
Ministerial N.413-93-TCC/15.01), el cual fue actualizado conforme la Resolución
Ministerial 210-2000 MTC/15.02, documento \citep{MTC:MDCTACC} donde se
especifican las características que un letrero de señalización debe de cumplir
para ser puesto en una vía peatonal y/o de tránsito vehicular.}

Uno de los enfoques, que ultimamente ganó mucho valor, fue aquel presentado en 
la Conferencia CPVR (Visión Computacional y Reconocimiento de Patrones - 
\textit{Computer Vision and Pattern Recognition}) del 2010
\footnote{\url{http://cvl.umiacs.umd.edu/conferences/cvpr2010/}} que obtuvo
muy buenos resultados bajo \textit{data sets} de competencias de ediciones 
anteriores. El SWT (\textit{Stroke Width Transform}) obtuvo resultados que 
variaban conforme a la imagen que se procesaba.

La característica peculiar de los letreros (mencionada anteriormente), hace que 
muchas de las aplicaciones de reconocimiento de texto fallen en casos donde el 
estado de los letreros se vea afectado por factores no usuales a los presentados
normalmente en fotografías de otra naturaleza (placas de automóviles, cubiertas 
de libros, señales de tránsito, anuncios publicitarios). 

\subsection{Identificación del Problema}
La tarea de segmentación de texto en fotografías que contienen letreros
informativos - direccionales de la ciudad de Cusco, hace que los resultados del
SWT, un enfoque segmentación de texto considerado como uno de los mejores, 
dependan en gran medida de la calidad de la imagen y el estado de cada letrero, 
llegando a fallar en muchos de los casos de prueba.

\section{Antecedentes}
``\textbf{Detecting Text in Natural Scenes with Stroke Width Transform}'' Boris
Epshtein, Eyal Ofek, Yonatan Wexler. Microsoft Corporation,
2010.\cite{Epshtein:SWT:2010}
\begin{itemize}
	\item En el trabajo, se muestra una técnica para incrementar los resultados
bajo la idea de \textit{stroke width} para la detección de texto.
	\item Se define la noción de \textit{stroke} y se deriva a un algoritmo
eficiente para calcularlo, produciéndose una nueva característica
(\textit{feature}) para la imagen. La nueva característica provee información
importante que fue demostrada para la detección de texto.
	\item El SWT (\textit{Stroke Width Transform}) combina estimaciones sobre cada
\textit{pixel} con método no locales de procesamiento de imágenes.
	\item Durante la fase de análisis de resultados, el algoritmo alcanzó el
primer lugar ejecutándose 15 veces más rapido que los otros algoritmos
reportados.
\end{itemize}

``\textbf{Scene Text Detection via Stroke Width}'' Yao Li y Huchuan Lu. School
of Information and Communication Engineering, Dalian University of Technology,
China, 2012. \cite{li:2012:scene}
\begin{itemize}
	\item En el trabajo, se presenta una nueva metodología basada en componentes
conexas para la detección de texto en imágenes con escenas naturales mediante la
aplicación del \textit{topological skeleton} como parte de un análisis de
siluetas en figuras para determinar el \textit{stroke width}.
	\item El método para agrupar componentes conexas no solo agrupa
caracteres en palabras, sino también, elimina falsos positivos.
\end{itemize}

``\textbf{Text Detection Nokia N900 Using Stroke Width Transform}'' Sauvar Kumar
y Andrew Perrault. Cornell University, Estados Unidos,
2010.\cite{Saurav:SWT:2010}
\begin{itemize}
	\item Se implementa una aplicación móvil para la segmentación de texto
mediante el enfoque del SWT presentando heurísticas robustas para el filtrado 
de \textit{strokes} y eliminación de falsos positivos.
	\item Se presenta una solución al problema de trabajar sobre imágenes con
texto claro y oscuro en una misma imagen mediante una combinación de resultados
para ambos casos.
	\item Son especificadas las reglas heurísticas usadas dentro del proceso de
filtrado de componentes basados en la distribución de color y la media aritmética
del \textit{stroke}.
\end{itemize}

\section{Objetivos}
  \subsection{Objetivo General}
  Proponer un enfoque para optimizar la segmentación de texto en fotografías que
contengan letreros informativos - direccionales de las calles de la ciudad de
Cusco haciendo uso del \textit{K-means} y mejorar los resultados obtenidos por 
el SWT.
  \subsection{Objetivos Específicos}
  \begin{enumerate}
  	\item Recolectar fotografías que contengan letreros informativos -
direccionales de las calles de la ciudad de Cusco.
    \item Investigar e implementar operaciones de transformación sobre imágenes
para su procesamiento.
    \item Investigar e implementar algoritmos para el proceso de segmentación de
texto en imágenes.
    \item Analizar y comparar los resultados obtenidos con el SWT.
  \end{enumerate}

\section{Justificación}
	Luego de someter el algoritmo SWT a la tarea de segmentación de texto bajo un
conjunto de 92 fotografías que contienen letreros informativos - direccionales
de las calles de Cusco, se obtuvo alredor de un $35\%$ de palabras corréctamente
segmentadas. Dado que las imágenes pueden ser grandes con respecto a sus
dimensiones, enfoques que desplacen ventanas por toda la imagen no son las más
adecuadas para la tarea de segmentación. 
 
	Por otra parte, no se cuenta con un algoritmo específico para la segmentación
del tipo de letreros considerados en el problema de investigación.

\section{Alcances}
  La captura de datos y construcción del \textit{data set} para fase de
entrenamiento y de prueba requeriría capturar imágenes con nombres de calles de
la ciudad de Cusco. Por la demanda de tiempo que significa todo ese proceso,
solo se usará un número limitado de elementos para la elaboración del 
\textit{data set}.

  El eje principal del proyecto es el desarrollo de procedimientos que hagan
posible la segmentación de texto en fotografías que contengan letreros de
señalización del tipo informativos y/o direccionales. El desarrollo de una
aplicación final de usuario que tenga en consideración aspectos relacionados a
interfaces y métodos de entrada, no está considerado como prioritario en el
presente trabajo.
  
  Pese a que se puede tomar en consideración ciertas condiciones que una
fotografía debería de cumplir para ser procesada, el interés del proyecto es la
segmentación de texto en imágenes influenciados bajo diversos factores que hagan
dificil el proceso de segmentación.
  
\section{Metodología}
  \subsection{Método de investigación}
  Dada la naturaleza del trabajo de tesis, se utiliza el método de
investigación teórica\footnote{Según la clasificación realizada por Elías
Mejía\citep{Mejia:2005:MIC}.} para un mejor entendimiento de los conceptos
asociados a la solución del problema planteado. Así como, los métodos 
descriptivo y experimental\footnote{De acuerdo a los tipos de
estudio en la investigación descrita en \citep{Hernandez:1999:MI}.} para buscar
las características que ayuden a mejorar el proceso de segmentación de texto.
Finalmente, se hace uso de la observación y comparación de resultados.

\section{Cronograma de actividades}
  \begin{figure}[h!]
    \centering
	  \includegraphics[angle=90, scale=0.8]{/Cap:AspectosGenerales/cronograma}
	  \caption{Cronograma de actividades}
	  \label{Fig:Cap:AspectosGenerales:Cronograma}
	\end{figure}
