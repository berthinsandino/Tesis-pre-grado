\chapter*{RECOMENDACIONES}
\label{cap:trabajosfuturos}
\addcontentsline{toc}{chapter}{\numberline{}RECOMENDACIONES}
\begin{enumerate}
	\item Se deja como trabajos futuros el desarrollo de las dos etapas para 
	resolver el problema de reconocimiento de texto, la segmentación de texto en
caracteres y el reconocimiento optico de caracteres. De igual forma, el enfoque
propuesto presenta una variante del algoritmo no supervisado \textit{K-means}
para resolver el problema de la clusterización, sin embargo hay otros algoritmos
como los mapas auto-organizativos de Kohonen y sus variantes
	
	\item Aunque, durante la fase de análisis de resultados hubo un intento de 
	plantear una hipótesis indicando un porcentaje de palabras corréctamente
segmentadas, para demostrarla hacía falta determinar la muestra en función al
tamaño de la población de letreros informativos - direccionales. Se deja al
lector la posibilidad de realizar un estudio más a fondo para determinar una
hipótesis y demostrarla. 
	
	\item Durante la fase de pre-procesamiento se hizo uso del algoritmo de 
	Niblack con un \textit{kernel} de $50 \times 50$, las dimensiones del mismo
fueron establecidas en base al enfoque de prueba y error, tal como se hizo para
determinar los valores de $\varphi$. Métodos estocásticos podrían determinar los
valores de dichos parámetros dependiendo de las características presentadas por
una fotografía.

	\item La realidad aumentada y el uso de dispotivos móviles representaría la 
	puerta a una futura aplicación atractiva para un usuario final en la cual se
pueda brindar información extra relacionada a las calles, como es el caso de la
aplicación móvil SappeAR.\cite{Sappear:web}
\end{enumerate}
