%% ------------------------------------------------------------------------- %%
%\chapter*{Abstract}%
\chapter*{INTRODUCCIÓN}
\label{cap:introduccion}
\addcontentsline{toc}{chapter}{\numberline{}INTRODUCCIÓN}%

El avance vertiginoso de la Informática permite experimentar algoritmos que 
antes solo eran enfocados de manera teórica. Las redes bayesianas y los 
algoritmos evolutivos son algunos ejemplos donde la complejidad computacional
era un factor crítico hasta hace unos años. Actualmente, se realizan trabajos 
basados en dichos algoritmos dejando de lado restricciones en cuanto a
su complejidad. Sin embargo, como producto del avance tecnológico, el 
incremento de los medios de comunicación (redes sociales y \textit{blogs}) trajo 
como consecuencia el hecho de tener una gran cantidad de información. Por 
ejemplo, las imágenes (en especial las fotografías), son un tipo de información 
que presentan detalles que solo en algunos casos son tomados en cuenta. Las 
imágenes satelitales, médicas y militares son solo una muestra de ello donde, 
la intepretación de ciertas características que presenta la imagen contribuyen 
al proceso de toma de decisiones.
%, por ende, se viene desarrollando y mejorando algoritmos que ayuden a una mejor interpretación.

Otro ejemplo puede darse con imágenes donde el objetivo principal sea el 
reconocimiento de texto, como fotografías que contengan letreros con nombres de 
las calles. Dado que, para el reconocimiento de texto, se hace uso de una fase
de segmentación para separar aquellos \textit{pixels} que formen parte del texto
(\textit{foreground}) de los \textit{pixels} que formen parte del 
\textit{background}. Por lo tanto, para el desarrollo del presente trabajo, se 
tomó en consideración el problema de la segmentación de texto en letreros de 
señalización (informativos y/o direccionales) de las calles de la ciudad de Cusco. 
 
Como resultado final, se propone un enfoque el cual se divide en tres etapas. La
primera etapa, comprende una serie de procedimientos de filtrado cuyo objetivo 
será descartar regiones de la imagen cuya información no sea relevante para el
desarrollo del trabajo. Para lo cual, se hará uso de métodos y técnicas de
procesamiento de imágenes. La segunda etapa, tiene como finalidad formar 
componentes conexas que serán pasadas como parámetro de entrada al algoritmo 
\textit{k-means clustering}, de esta forma, se podrá disminuir el número de 
operaciones del \textit{k-means}. Finalmente, la tercera y última etapa hace uso de una 
serie de reglas heurísticas con la finalidad de filtrar regiones de la imagen 
formadas en base a un subconjunto de \textit{clusters} (regiones que sean muy 
poco probables de formar parte del texto).

Para tal efecto, se vio por conveniente enfocar su desarrollo en tres partes:

\begin{description}
	\item[Parte I] Enfocado nétamente a cubrir los aspectos generales del problema.
  \item[Parte II] Involucra aspectos relacionados al marco teórico.
	\item[Parte III] Desarrolla el enfoque propuesto, así como las conclusiones y 
	recomendaciones.
\end{description}
