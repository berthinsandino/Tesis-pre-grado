%% ------------------------------------------------------------------------- %%
%\chapter*{Abstract}%
\chapter*{INTRODUCCIÓN}
\label{cap:introduccion}
\addcontentsline{toc}{chapter}{\numberline{}INTRODUCCIÓN}%

El avance vertiginoso de la Informática permite experimentar algoritmos que 
antes solo eran enfocados de manera teórica. Las redes bayesianas y los 
algoritmos evolutivos son algunos ejemplos, donde la complejidad computacional
era un factor crítico hasta hace unos años. En la actualidad, se realizan
trabajos basados en dichos algoritmos dejando de lado restricciones en cuanto a
su complejidad. 

Sin embargo, como producto del avance tecnológico, el incremento de los medios
de comunicación (redes sociales y \textit{blogs}) trajo como consecuencia mucha
información. Por ejemplo, las imágenes (en especial las fotografías), son un
tipo de información que presentan detalles que solo en algunos casos son tomados
en cuenta. Las imágenes satelitales, médicas y militares son solo una muestra de
ello donde, la intepretación de ciertas características que presenta la imagen
contribuyen al proceso de toma de decisiones.%, por ende, se viene desarrollando
y mejorando algoritmos que ayuden a una mejor interpretación.

El presente trabajo, tiene como punto de interés la segmentación de texto en 
letreros de señalización (informativos y/o direccionales) de las calles de la
ciudad de Cusco. En el desarrollo del trabajo de investigación se consideraron
tres puntos importantes.

El primer punto, comprende una serie de procedimientos de filtrado cuyo objetivo
será descartar regiones de la imagen cuya información no sea relevante para el
desarrollo del trabajo. Para lo cual, se hará uso de métodos y técnicas de
procesamiento de imágenes. 

El segundo punto, tiene como finalidad formar componentes conexas que serán
pasadas como parámetro de entrada al algoritmo \textit{k-means clustering}, de
esta forma, se podrá disminuir el número de operaciones del \textit{k-means}.

El tercer y último punto hace uso de una serie de reglas heurísticas con la 
finalidad de filtrar regiones de la imagen formada en base a un subconjunto de
\textit{clusters}, regiones que sean muy poco probables de formar parte del
texto.

Para tal efecto, se ha visto por conveniente enfocar su desarrollo en tres partes:
\begin{description}
	\item[Parte I] Enfocado nétamente a cubrir los aspectos generales.
  \item[Parte II] Enfocado al marco teórico.
	\item[Parte III] Desarrollo del enfoque propuesto, conclusiones y recomendaciones.
\end{description}
