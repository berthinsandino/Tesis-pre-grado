\chapter*{CONCLUSIONES}
\label{cap:conclusiones}
\addcontentsline{toc}{chapter}{\numberline{}CONCLUSIONES}
\begin{enumerate}
	\item Se propuso un nuevo enfoque para la segmentación de texto en fotografías
	que contengan letreros informativos - direccionales de las calles de la ciudad
de Cusco mediante variaciones del algoritmo \textit{K-means clustering} y la
estructura de datos \textit{Disjoint-sets}, lográndose optimizar los resultados.
	\item Se recolectó un conjunto de 120 fotografías\footnote{Todas las 
	fotografías se encuentran en el disco adjunto al ejemplar de tesis.}
conteniendo letreros informativos - direccionales de la ciudad de Cusco, de las
cuales se armó un \textit{data set} de 92 fotografías con el objetivo de abordar
aspectos que influyan en los resultados como el estado del letrero, inclinación,
tamaño, calidad de la fotografía, niveles de luz y oscuridad (sombras).
	\item Todos los algoritmos presentados fueron implementados haciendo uso de 
	diferentes lenguajes de programación. Sin embargo, se utilizó C++ como
lenguaje principal sobre el cual, fueron implementados únicamente los algoritmos
usados en el enfoque propuesto optimizándolos con la finalidad de dismiuir su
complejidad. \footnote{Para consultar detalládamente las implementaciones, hacer
referencia del código fuente adjunto en el disco o el repositorio
\url{https://github.com/berthinsandino/Tesis-pre-grado/tree/master}.}
		\begin{itemize}
			\item Estructuras de datos:
			\begin{itemize}
				\item \textit{Hash table}. Se implementó un \textit{hash table} de 
				tamaño $2^b$ para elementos con $b$ \textit{bits} de longitud, de tal
				forma que, para un valor $k$, el $hash_m(k)$ está definido como:
				\begin{equation}
					hash_m(k) = k \text{ xor } 2^b - 1
				\end{equation}
				\item \textit{Disjoint sets}. Estructura que hace uso del algoritmo 
				\textit{Union-Find} con las optimizaciones de los enfoques
				\textit{weighted quick union} y \textit{path compression}.
			\end{itemize}
			\item Algoritmos:
			\begin{itemize}
			\item Filtro \textit{ad-hoc}. Hace uso de un enfoque de conversión a 
			enteros para hallar el valor \textit{Hue} evitándose el uso de operaciones
			punto flotante.
			\item \textit{Median filter}. Emplea un método óptimo de búsqueda de la 
			mediana para un kernel con $S=3$. Lleva a cabo 19 operaciones de
			intercambio y cada intercambio hace uso de tres operaciones \textbf{xor},
			de esta forma se logra evitar el uso de memoria adicional.
			\item \textit{Niblack's thresholding}. Para el cálculo de los estadísticos
			$\mu$ y $\sigma^2$. \footnote{Para determinar el valor de
			$\sigma$(desviación estándar), primero se debe determinar el valor de
			$\sigma^2$(varianza).} Se utilizaron dos tablas aditivas, por lo que en
			$O(1)$ se puede llegar a determinar dichos valores para un kernel de
			tamaño $K$ cualquiera.
			\item \textit{K-means clustering}. Se presentó una solución polinomial a 
			un problema no polinomial.
			\item \textit{Refinement step} (regla 3). Mediante el uso de el enfoque de
			compresión de coordenadas y tablas aditivas se determinó en $O(2\lg(c))$
			el número de componentes dentro de una componente, siendo $c$ el número
			total de componentes.
			\end{itemize}
		\end{itemize}
	\item El enfoque propuesto hace uso de los algoritmos \textit{Union - Find}
	(UF) y \textit{K-means clustering}(KM). El UF reduce los parámetros de entrada
	para el KM. Mientras que, el KM, es usado para resolver el problema de
	clusterización mediante una solución polinomial (P) limitando el número de
	iteraciones.\footnote{Al proponerse la solución P, las soluciones brindadas
	por el algoritmo pueden o no llegar al \textit{global optimum}.}
	\item El enfoque propuesto, hace uso de un método no supervisado de 
	aprendizaje para segmentar imágenes con una característica peculiar, el color.
	También, se presenta una forma de combinar métodos locales y globales para el
	procesamiento de imágenes. Locales cuando se usó la información local
	(\textit{neighbors}) para crear componentes basándose en principios de
	proximidad del color, y globales durante la fase de \textit{clustering}. El
	enfoque es capaz de operar bajo limitaciones de memoria en comparación con el
	enfoque SWT, y aunque SWT es más rápido en tiempo de ejecución, se obtuvieron
	mejores resultados.	  
\end{enumerate}
